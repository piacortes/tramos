%% Copyright (C) 2016 Patricio Rojo, Carolina Gutierrez

%% This program is free software; you can redistribute it and/or
%% modify it under the terms of version 2 of the GNU General 
%% Public License as published by the Free Software Foundation.

%% This program is distributed in the hope that it will be useful,
%% but WITHOUT ANY WARRANTY; without even the implied warranty of
%% MERCHANTABILITY or FITNESS FOR A PARTICULAR PURPOSE.  See the
%% GNU General Public License for more details.

%% You should have received a copy of the GNU General Public License
%% along with this program; if not, write to the Free Software
%% Foundation, Inc., 51 Franklin Street, Fifth Floor, 
%% Boston, MA  02110-1301, USA.



\documentclass{article}

%% README
%%
%% Before running pdflatex in this file, macro names like
%% @MACRO@ should be replaced by their real VALUE. This
%% could be accomplished by using sed, for instance:
%%
%% sed -e 's/@PLANET@/Wasp-43/g; s/@EPOCH@/20160115T23:00/g; s/@RA@/00:00/g; s/@DEC@/-01:00/g;' <extraction.tex >extraction_wasp43-20160115T23_00.tex
%%
%%  And then, run pdflatex twice


\usepackage[utf8]{inputenc}
\usepackage{geometry}

\newgeometry{left=3cm, right=3cm}

\usepackage{graphicx}
%%\usepackage{lastpage}
\usepackage{fancyhdr}
\pagestyle{fancy}
\usepackage{tramos_custom}
\usepackage{listings}


\chead{\footnotesize \planet: \epoch}
\rhead{\footnotesize page \thepage}
\cfoot{}

\newenvironment{cfigure}{
  \begin{figure}
    \begin{center}
}
{
    \end{center}
  \end{figure}
}



\begin{document}

\begin{center}
\begin{Large}
  TRAMOS Reduction Pipeline: \planet
\end{Large}

\vfill

\begin{tabular}{rl}
  \hline\hline
  epoch & \epoch \\
  telescope/instrument & \inst \\
  period & \period \\
  semi-major axis & \sma \\
  RA & \ra \\
  Dec & \dec \\
  V  & \magv \\
  Chosen Apperture & \aperture \\
  \hline\hline
\end{tabular}
                 \end{center}
\vfill

Reduction notes: \comments
\vfill\vfill

\begin{figure}[h]
  \centering
  \includegraphics[height=0.5\textheight]{figures/ratio.png}
  \caption{Flux ratio of the target. the x-axis it is in Julian Date.}
\end{figure}

\clearpage

\begin{center}
  \includegraphics[height=0.4\textheight]{figures/ref.png}
\end{center}
\begin{cfigure}
  \includegraphics[height=0.4\textheight]{figures/stamp.png}
  \caption{Timeseries stamps for target planet}
\end{cfigure}

\clearpage

\begin{cfigure}
  \includegraphics[height=0.4\textheight]{figures/radialprof.png}
  \caption{Radial profile with the range of apertures and the final aperture (optimal) indicated.}
\end{cfigure}
\begin{cfigure}
  \includegraphics[width=0.45\textwidth]{figures/exptime.png}
  \includegraphics[width=0.45\textwidth]{figures/centers_xy.png}
  \caption{plot of variations of exposure time (left) and
  variations of positions for centers (right) for target and references,
  the variation is shown in $\Delta x = x_i - x_0$ and $\Delta y = y_i - y_0$ using position
  (x,y) of first frame as reference ($x_0, y_0$).}
\end{cfigure}

\clearpage

\begin{cfigure}
  \includegraphics[height=0.4\textheight]{figures/apert.png}
  \caption{Flux out of transit for diferent apertures in a large range. \label{flux1}}
\end{cfigure}
\begin{cfigure}
  \includegraphics[height=0.35\textheight]{figures/apertstd.png}
  \caption{Plot of aperture vs standard deviation and mean errorbar for
  the diferent apertures used in Fig.\ \ref{flux1}}
\end{cfigure}

\clearpage

\begin{cfigure}
  \includegraphics[height=0.4\textheight]{figures/apert2.png}
  \caption{Flux out of transit for different apertures in a small range. \label{flux2}}
\end{cfigure}
\begin{cfigure}
  \includegraphics[height=0.35\textheight]{figures/apertstd2.png}
  \caption{Plot of aperture vs standard deviation and mean errorbar for
  the diferent apertures used in Fig.\ \ref{flux2}. \label{flux2std}}
\end{cfigure}

\clearpage

\begin{cfigure}
  \includegraphics[height=0.45\textheight]{figures/sky.png}
  \caption{gráfico flujo fuera de tránsito a distintos anillos manteniendo la misma área.\label{sky}}
\end{cfigure}
\begin{cfigure}
  \includegraphics[width=0.45\textwidth]{figures/skystd.png}
  \includegraphics[width=0.45\textwidth]{figures/skystdarea.png}
  \caption{gráfico cielo (left) razon de area(right) vs desviación estándar y barra de error media para los anillos de cielo en Fig.\ \ref{sky} (con el radio interno en el eje x).}
\end{cfigure}

\clearpage

\begin{cfigure}
  \includegraphics[height=0.45\textheight]{figures/sky_outradius.png}
  \caption{Flux out of transit for different sky external radius with the same internal radius (chosen in a arbitrary way).\label{sky_inrad}}
\end{cfigure}
\begin{cfigure}
  \includegraphics[width=0.45\textwidth]{figures/sky_outradiusstd.png}
  \includegraphics[width=0.45\textwidth]{figures/sky_outradiusstdarea.png}
  \caption{Plot sky (left) and area ratio (right) vs standard deviation and mean errorbar for the skys used in Fig.\ \ref{sky_inrad}.\label{sky_inradstd}}
\end{cfigure}

\clearpage

\begin{cfigure}
  \includegraphics[height=0.45\textheight]{figures/sky_inradius.png}
  \caption{Flux out of transit for different sky internal radius with the same
  external radius (chosen from the minimun standard deviation in Fig.\ \ref{sky_inradstd}).\label{sky_outrad}}
\end{cfigure}
\begin{cfigure}
  \includegraphics[width=0.45\textwidth]{figures/sky_inradiusstd.png}
  \includegraphics[width=0.45\textwidth]{figures/sky_inradiusstdarea.png}
  \caption{Plot sky (left) and area ratio (right) vs standard deviation and mean errorbar for the skys used in Fig.\ \ref{sky_outrad}}
\end{cfigure}

\clearpage

\begin{cfigure}
  \includegraphics[height=0.4\textheight]{figures/normflux.png}
  \caption{Normalized flux of the target and references at the
  chosen aperture (from the minimum standard deviation in Fig.\ \ref{flux2std}) for the entire time series, the x-axis it is in Modified Julian Date.}
\end{cfigure}
\begin{cfigure}
  \includegraphics[height=0.4\textheight]{figures/flux.png}
  \caption{Flux of the target and references at the chosen aperture (from the minimum standard deviation in Fig.\ \ref{flux2std}) for the entire time series, the x-axis it is in Modified Julian Date.}
\end{cfigure}

\clearpage

\begin{cfigure}
  \includegraphics[height=0.45\textheight]{figures/mom2_mag.png}
  \caption{Magnitude of second momentum for target and references.}
\end{cfigure}
\begin{cfigure}
  \includegraphics[width=0.46\textwidth]{figures/peak.png}
  \includegraphics[width=0.43\textwidth]{figures/excess.png}
  \caption{Plot of variations of the peak (left) and excess of pixels (right) relative to
  the chosen aperture (from the minimum standard deviation in Fig.\ \ref{flux2std}).}
\end{cfigure}

\clearpage

\begin{cfigure}
  \includegraphics[height=0.45\textheight]{figures/fwhm.png}
  \caption{Variations of the full width half maximum across the time series for
  target and references.}
\end{cfigure}
\begin{cfigure}
  \includegraphics[width=0.45\textwidth]{figures/mom3_mag.png}
  \includegraphics[width=0.45\textwidth]{figures/mom3_ang.png}
  \caption{Variations of magnitude(left) and angle (right) of the third momentum for
  target and references.}
\end{cfigure}

\clearpage

\vspace{20cm}
Example of running the pipeline:
\\
\pipeline

\end{document}


